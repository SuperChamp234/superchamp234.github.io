\documentclass[letterpaper,11pt]{article}

\usepackage{latexsym}
\usepackage[empty]{fullpage}
\usepackage{titlesec}
\usepackage{marvosym}
\usepackage[usenames,dvipsnames]{color}
\usepackage{verbatim}
\usepackage{enumitem}
\usepackage[hidelinks]{hyperref}
\usepackage{fancyhdr}
\usepackage[english]{babel}
\usepackage{tabularx}
\input{glyphtounicode}
\usepackage{setspace}
\usepackage{multicol}
\RequirePackage{fontawesome}
%\usepackage{parskip}
\usepackage{lipsum}

%----------FONT OPTIONS----------
% sans-serif
% \usepackage[sfdefault]{FiraSans}
% \usepackage[sfdefault]{roboto}
% \usepackage[sfdefault]{noto-sans}
% \usepackage[default]{sourcesanspro}

% serif
% \usepackage{CormorantGaramond}
% \usepackage{charter}


\pagestyle{fancy}
\fancyhf{} % clear all header and footer fields
\fancyfoot{}
\renewcommand{\headrulewidth}{0pt}
\renewcommand{\footrulewidth}{0pt}

% Adjust margins
\addtolength{\oddsidemargin}{-0.6in}
\addtolength{\evensidemargin}{-0.5in}
\addtolength{\textwidth}{1in}
\addtolength{\topmargin}{-.7in}
\addtolength{\textheight}{1in}


\urlstyle{same}

\raggedbottom
\raggedright
\setlength{\tabcolsep}{0in}

% Sections formatting
\titleformat{\section}{
  \vspace{-6pt}\scshape\raggedright\large\bfseries
}{}{0em}{}[\color{black}\titlerule \vspace{-6pt}]

% Ensure that generate pdf is machine readable/ATS parsable
\pdfgentounicode=1

%-------------------------
% Custom commands
\newcommand{\resumeItem}[1]{
  \item\small{
    {#1 \vspace{-2pt}}
  }
}

\newcommand{\classesList}[4]{
    \item\small{
        {#1 #2 #3 #4 \vspace{-2pt}}
  }
}

\newcommand{\resumeSubheading}[4]{
  \vspace{-2pt}\item
    \begin{tabular*}{1.0\textwidth}[t]{l@{\extracolsep{\fill}}r}
      \textbf{#1} & \textbf{\small #2} \\
      \textit{\small#3} & \textit{\small #4} \\
    \end{tabular*}\vspace{-7pt}
}

\newcommand{\resumeSubSubheading}[2]{
    \item
    \begin{tabular*}{0.97\textwidth}{l@{\extracolsep{\fill}}r}
      \textit{\small#1} & \textit{\small #2} \\
    \end{tabular*}\vspace{-7pt}
}

\newcommand{\resumeProjectHeading}[2]{
    \item
    \begin{tabular*}{1.001\textwidth}{l@{\extracolsep{\fill}}r}
      \small#1 & \textbf{\small #2}\\
    \end{tabular*}\vspace{-7pt}
}

\newcommand{\resumeSubItem}[1]{\resumeItem{#1}\vspace{-5pt}}

\renewcommand\labelitemi{$\vcenter{\hbox{\tiny$\bullet$}}$}
\renewcommand\labelitemii{$\vcenter{\hbox{\tiny$\bullet$}}$}

\newcommand{\resumeSubHeadingListStart}{\begin{itemize}[leftmargin=0.0in, label={}]}
\newcommand{\resumeSubHeadingListEnd}{\end{itemize}}
\newcommand{\resumeItemListStart}{\begin{itemize}}
\newcommand{\resumeItemListEnd}{\end{itemize}\vspace{-5pt}}

\makeatletter
\newcommand{\github}[1]{%
   \href{#1}{\faGithubSquare}%
}
\makeatother

\setlist[itemize]{ itemsep=2pt, topsep=3pt}

\begin{document}

\begin{center}
  \textbf{\Huge \scshape Zain Siddavatam} \\ \vspace{1pt}
  \faEnvelope~\href{mailto:hi@zainsv.me}{hi@zainsv.me} $|$
  \faLinkedin~\href{https://www.linkedin.com/in/zain-siddavatam/}{/zain-siddavatam} $|$
  \faGithub~\href{https://github.com/SuperChamp234}{/SuperChamp234} $|$
  \faPhone~\href{tel:+917506465276}{+91 7506465276} $|$
  \faGlobe~\href{https://zainsv.me}{zainsv.me}
\end{center}

%-----------EDUCATION-----------
%vspace{-1pt}
\section{Education}
\resumeSubHeadingListStart
\resumeSubheading
{Veermata Jijabai Technological Institute}{December 2021 -- Present}
{B.Tech Student in Electrical Engineering}{Mumbai}
\resumeItemListStart
\resumeItem{\textit{Relevant coursework: Analog and Digital Circuits 9/10, Electronic Devices and Circuits 10/10, Control Systems 9/10, Microprocessors and Microcontrollers 10/10}.}
\resumeItemListEnd
\resumeSubHeadingListEnd

%----------EXPERIENCES---------

\section{Experiences}
\resumeSubHeadingListStart
\resumeSubheading
{Embedded Software Engineer Intern at Beagleboard.org \href{https://openbeagle.org/superchamp234}{\faExternalLink}}{}{Verilog HDL, TCL, Polarfire SoCs, ARM AMBA, C/C++}{June 2024 - September 2024}
\resumeItemListStart
\resumeItem{Created CI workflow for generating custom Debian GNU/Linux images for the \textbf{BeagleY-AI} single-board computer.}
\resumeItem{Contributed to the documentation and on-boarding examples for \textbf{BeagleV-Fire} single-board computer.}
\resumeItem{Added support for industrial communication addon board for the BeagleV-Fire thus allowing \textbf{CAN}, \textbf{RS485}, and \textbf{4-20 mA current loop} peripherals.}
\resumeItem{Created examples demonstrating the usage of \textbf{AXI4} and \textbf{APB3 peripherals} on the BeagleV-Fire's \textbf{PolarFire MPFS025T SoC}. These examples enable users to control peripherals from Linux using C/C++.}
\resumeItemListEnd
\resumeSubheading
{Hardware Acceleration Intern at Acceleration Robotics \href{https://drive.google.com/file/d/1zK9jAKZ9oBVwV1-_GcFWSXyqZOX0cpnZ/view?usp=sharing}{\faExternalLink}}{}{C++/C, Vitis HLS, Vivado, KV260, Kria Robotics Stack}{January 2024 - June 2024}
\resumeItemListStart
\resumeItem{Accelerated \textbf{ROS2} workloads by optimizing \textbf{Adaptive Monte-Carlo Localization (AMCL)}.}
\resumeItem{Designed and deployed a ROS2 node with a kernel for an accelerated particle filter algorithm on the \textbf{KV260} platform.}
\resumeItem{Developed a custom KV260 platform in Vivado for HLS kernel deployment, achieving a \textbf{1.95x speedup} over the onboard SoC's CPU.}
\resumeItemListEnd
\resumeSubheading
{Research Intern at CASL, University of Maryland  \href{{https://drive.google.com/drive/folders/19r-r_VaqXl1FvfySPe8bHm7v4QKsKUU0?usp=sharing}}{\faExternalLink}}{}
{C++/C, Vitis HLS, Vivado}{February 2023 - October 2023}
\resumeItemListStart
\resumeItem{Researched Domain-Specific Architectures, focusing on dataflows for Sparse Matrix General Matrix Multiplication (SpGEMM).}
\resumeItem{Conducted a comprehensive survey of prevalent techniques used in SpGEMM.}
\resumeItem{Implemented two sparse matrix multiplication algorithms using \textbf{Vitis High-Level Synthesis (HLS)}: an \textbf{Outer Product dataflow} and a \textbf{Row Product dataflow}.}
%\resumeItem{Executed optimization techniques such as pipelining, unrolling, and using specialized hls libraries to significantly enhance the efficiency of the sparse matrix multipliers.}
\resumeItemListEnd
\resumeSubHeadingListEnd

%-----------PROJECTS-----------
\section{Projects}
\resumeSubHeadingListStart
\resumeSubheading
{Hackathon: RISC-V Softmax Microcode (RISC-V Intl. x Andes) \href{https://docs.google.com/presentation/d/15_osaoplCDb2VvI0H95oHUXygs7DukYd/edit?usp=sharing}{\faExternalLink}}{}
{RISC-V Spike, C, Inline Assembly, Performance Optimization}{April 2025}
\resumeItemListStart
\resumeItem{Implemented a custom \textbf{softmax instruction} for RISC-V CPUs, using \textbf{inline assembly} and \textbf{C} for microcode generation.}
\resumeItem{Increased signal-to-noise ratio and improved execution speed by \textbf{50\%} using a custom \textbf{exp function} via \textbf{Estrin’s scheme}.}
\resumeItem{Awarded the \textbf{Favourite Implementation Prize} for performance improvements and approach.}
\resumeItemListEnd

\resumeSubheading
{32-bit RISC-V CPU Design \& Verification \href{{https://github.com/sra-vjti/synapse32}}{\faExternalLink}}{}
{Verilog HDL, Cyclone II, Quartus Prime}{August 2022, May 2025}
\resumeItemListStart
\resumeItem{Implemented a \textbf{5-stage RV32I-compliant} CPU core in Verilog and deployed it on a Cyclone II FPGA with seven-segment display output.}
\resumeItem{Built a custom assembler for RV32I instructions and ran example programs like Fibonacci sequence.}
\resumeItem{Verified the CPU using Cocotb with Python-based testbenches and instruction tests from the official RISC-V ISA test suite.}
\resumeItem{Developed monitors, scoreboards, and self-checking test cases to catch bugs in ALU ops, branching, etc.}
\resumeItemListEnd

\resumeSubheading
{WiFi Controlled Bot \href{{https://github.com/SuperChamp234/wifi-bot-sra}}{\faExternalLink}}{}
{C/C++, ESP-IDF framework}{April 2022 - May 2022}
\resumeItemListStart
\resumeItem{Developed a \textbf{remote controlled bot} using ESP-32.}
\resumeItem{Used the \textbf{WiFi} onboard the ESP32 for relaying data from the car.}
\resumeItem{\textbf{FreeRTOS} is used to facilitate the task of moving as well as relaying data.}
\resumeItem{Implemented a web page for remote control, which can accept touch as well as keyboard inputs.}
\resumeItemListEnd
\resumeSubheading
{Habitica Sync
\href{https://github.com/SuperChamp234/Habitica-sync}{\faExternalLink}}{}
{Typescript, React, NodeJS}{2021}
\resumeItemListStart
\resumeItem{Developed a widget for the Obsidian note taking app, for allowing users to access their tasks and rewards from the widely used Habitica Todo-list application.}
\resumeItem{Designed a UI using react and material-ui, and integrated habitica's features using their API.}
\resumeItem{The widget received 1000+ downloads from Obsidian's plugin store.}
\resumeItemListEnd
\begin{samepage}
\resumeSubheading
{Google Code-in \href{{https://codein.withgoogle.com/archive/}}{\faExternalLink}}{}
{Java, CSS, HTML, Javascript}{2017, 2018}
\resumeItemListStart
\resumeItem{Mentee in Google Code-In contest, mentored by open source companies. {\href{{https://codein.withgoogle.com/archive/search/?q=Zain\%20Siddavatam}}{\faExternalLink}}}
\resumeItem{Developed a \textbf{terrain generator}, to generate different kinds of blocks at different rarities and vary terrain features at different heights. \href{https://codein.withgoogle.com/archive/2018/organization/5715161717407744/task/4740560489283584/}{\faExternalLink}.}
\resumeItem{Composed \textbf{two web pages} for BRL-CAD, an about page and an onepager website. \href{https://codein.withgoogle.com/archive/2017/task/4648191848873984/}{\faExternalLink} \href{https://codein.withgoogle.com/archive/2017/organization/6254981527109632/task/5406096005005312/}{\faExternalLink}.}
\resumeItemListEnd
\end{samepage}
\resumeSubHeadingListEnd

-----------COMPETITIONS-----------
\section{Competitions}
\resumeSubHeadingListStart
\resumeSubheading
{Competition: FPGA-based Maze Solving Robot \href{https://portal.e-yantra.org/}{\faExternalLink}}{}
{Verilog, Quartus, SPI/UART}{Sept 2022 – Jan 2023}
\resumeItemListStart
\resumeItem{Created a \textbf{maze-solving robot} on \textbf{Cyclone IV FPGA} using Verilog, implementing PID control and Dijkstra’s algorithm for pathfinding.}
\resumeItem{Integrated sensor data via \textbf{onboard ADCs}, \textbf{SPI}, and \textbf{UART} for communicating with the host via ZigBee.}
\resumeItem{Advanced to Stage 2 in the E-Yantra national-level robotics challenge.}
\resumeItemListEnd
\resumeSubHeadingListEnd

%-----------PROGRAMMING SKILLS-----------
\section{Technical Skills, Language Skills}
\begin{tabular}{ll}
  \textbf{Programming Languages:} & Verilog HDL, TL-Verilog, C/C++, Python, Java, JavaScript, TypeScript \\
  \textbf{Hardware Frameworks:}   & Quartus Prime, Vivado, Vitis HLS, ESP-IDF Toolchain                  \\
  \textbf{Software Frameworks:}   & Linux, OpenCV, Numpy, ReactJS, NodeJS, Git, CocoTB, UVM (familiar)   \\
  \textbf{Hardware Platforms:}    & Arty A7, Cyclone, iCE40, Zynq SoCs, Kria KV260                       \\
  \textbf{Languages:}             & English (Fluent)                                                   
\end{tabular}

%-----------ExtraCurricular-----------
\section{Extracurricular}
\resumeSubHeadingListStart
\resumeSubheading
{Joint General Secretary - Society of Robotics and Automation, VJTI}{}{Student-run robotics and automation club \href{https://sravjti.in}{\faExternalLink}}{May 2023 -- May 2024}
\resumeItemListStart
\resumeItem{Oversee the day-to-day management of the club, including strategic planning, event coordination, and team leadership for a 40-member strong team.}
\resumeItem{Successfully organized and conducted workshops for 200+ first-year students on cutting-edge topics, such as ESP-32 based Line following robots, Image Processing, and ROS based 3-DOF manipulators. These workshops provided hands-on experience and introduced newcomers to advanced robotics concepts.}
\resumeItem{Pioneered the Eklavya Mentorship Program, a two-month initiative, guiding three teams second-year students through FPGA-based projects.}
\resumeItemListEnd
\resumeSubHeadingListEnd

\end{document}
